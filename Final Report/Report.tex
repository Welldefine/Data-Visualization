\documentclass[12pt, a4paper, oneside]{ctexart}
\usepackage[hmargin=1in,vmargin=1in]{geometry}
\usepackage{amsmath, amsthm, amssymb, bm, color, framed, graphicx, hyperref, mathrsfs,extarrows}
\title{\textbf{数据可视化报告——北京市空气治理}}
\author{田润泽-2023201906}
\date{乙巳年腊月十五}
\linespread{1.5}
\usepackage{background}
\backgroundsetup{scale=0.4, angle=0, opacity = 1,contents = {\includegraphics[width=\paperwidth, height=\paperwidth, keepaspectratio]{logo-RUC.png}}}

\begin{document}
\maketitle
\begin{abstract}
    本文通过对北京市空气质量数据的分析,探讨了北京市空气质量的现状,以及北京市近年来空气治理的情况。通过对数据的处理和可视化,我们发现了一些有趣的现象,例如北京市空气质量的季节性变化,以及不同污染物之间的相关性。我们还通过对数据的分析,得到了空气质量的分布与变化,以帮助人们合理安排出行与出游。最后,我们总结了本文的研究结果,并提出了一些可能的研究方向。
\end{abstract}

\section{研究背景}
近年来,北京市作为中国的首都,其空气质量问题备受社会各界关注。由于地理位置、快速的城市化进程以及能源消费结构的影响,北京长期以来面临着空气污染的严峻挑战,尤其是PM2.5和PM10等颗粒物污染问题。这不仅对市民的健康造成了威胁,也对城市的可持续发展和国际形象产生了重要影响。

为应对这一问题,北京市政府及相关部门采取了一系列治理措施,包括调整能源结构、优化交通管理、推进工业减排和增加绿化覆盖率等。在国家政策的支持下,北京还积极参与了区域联防联控,与周边城市协同治理大气污染。这些努力已取得显著成效,例如2013年至今,北京的年均PM2.5浓度逐步下降,空气质量优良天数显著增加。

然而,尽管空气质量治理取得了一定进展,北京的空气污染问题仍然存在一些复杂性和不确定性。例如,冬季采暖季的污染反弹、区域传输效应和气象条件对空气质量的影响等,仍是当前研究和治理的难点。

本研究以北京市空气质量的治理与现状为主题,通过数据可视化技术,分析北京市空气质量的特点,近年来北京市空气质量的改善趋势、治理措施的成效。通过直观的可视化展示,不仅可以帮助公众更好地理解空气质量治理的重要性,也为政府决策和未来研究提供参考依据。
\section{研究目的}
本研究旨在通过数据可视化技术,全面分析北京市近年来空气质量的治理成果与现状。
\begin{enumerate}
    \item 展示变化趋势
    过对北京市近十年来空气质量数据(如PM2.5、PM10、SO2、NO2等指标)的分析,直观呈现空气质量的总体变化趋势,探究不同时间节点的治理成效与环境改善情况。
    \item 展示空气质量特征
    通过对北京市空气质量数据的分析,展示不同季节、不同区域的空气质量特征,探究空气质量的时空分布规律。
\end{enumerate}
\section{研究方法}
\section{数据来源}
\section{数据处理}
\section{数据可视化}
\section{结论}
\section{致谢}

\end{document}
