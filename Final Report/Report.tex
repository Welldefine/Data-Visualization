\documentclass[12pt, a4paper, oneside]{ctexart}
\usepackage[hmargin=0.7in,vmargin=0.7in]{geometry}
\usepackage{placeins}
\usepackage{amsmath, amsthm, amssymb, bm, color, framed, graphicx, hyperref, mathrsfs,extarrows}
\title{\textbf{数据可视化报告——北京市空气治理}}
\author{田润泽-2023201906}
\date{乙巳年腊月十五}
\linespread{1.5}
\usepackage{background}
\backgroundsetup{scale=0.4, angle=0, opacity = 1,contents = {\includegraphics[width=\paperwidth, height=\paperwidth, keepaspectratio]{logo-RUC.png}}}

\begin{document}
\maketitle
\begin{abstract}
    本文通过对北京市空气质量数据的分析,
    探讨了北京市空气质量的现状,以及北京市近年来空气治理的情况。
    通过对数据的处理和可视化,我们发现了一些有趣的现象,
    例如北京市空气质量的季节性变化,以及不同污染物之间的相关性。
    我们还通过对数据的分析,得到了空气质量的分布与变化,
    以帮助人们合理安排出行与出游。本报告的一切数据、
    代码与可视化成果均已同步至个人github平台:
    \url{https://github.com/Welldefine/Data-Visualization}
\end{abstract}

\section{研究背景}
近年来,北京市作为中国的首都,其空气质量问题备受社会各界关注。由于地理位置、快速的城市化进程以及能源消费结构的影响,北京长期以来面临着空气污染的严峻挑战,尤其是PM2.5和PM10等颗粒物污染问题。这不仅对市民的健康造成了威胁,也对城市的可持续发展和国际形象产生了重要影响。

为应对这一问题,北京市政府及相关部门采取了一系列治理措施,包括调整能源结构、优化交通管理、推进工业减排和增加绿化覆盖率等。在国家政策的支持下,北京还积极参与了区域联防联控,与周边城市协同治理大气污染。这些努力已取得显著成效,例如2013年至今,北京的年均PM2.5浓度逐步下降,空气质量优良天数显著增加。

然而,尽管空气质量治理取得了一定进展,北京的空气污染问题仍然存在一些复杂性和不确定性。例如,冬季采暖季的污染反弹、区域传输效应和气象条件对空气质量的影响等,仍是当前研究和治理的难点。

本研究以北京市空气质量的治理与现状为主题,通过数据可视化技术,分析北京市空气质量的特点,近年来北京市空气质量的改善趋势、治理措施的成效。通过直观的可视化展示,不仅可以帮助公众更好地理解空气质量治理的重要性,也为政府决策和未来研究提供参考依据。
\section{研究目的}
本研究旨在通过数据可视化技术,全面分析北京市近年来空气质量的治理成果与现状。
\begin{enumerate}
    \item 展示变化趋势
    过对北京市近十年来空气质量数据(如PM2.5、PM10、SO2、NO2等指标)的分析,直观呈现空气质量的总体变化趋势,探究不同时间节点的治理成效与环境改善情况。
    \item 展示空气质量特征
    通过对北京市空气质量数据的分析,展示不同季节、不同区域的空气质量特征,探究空气质量的时空分布规律。
\end{enumerate}
\section{数据说明与数据来源}
\textbf{数据来源:} \href{http://www.bjmemc.com.cn/}{北京市环境保护检测中心} (2013.12.06–2024.11.16) - \href{https://quotsoft.net/air/}{王晓磊}

\subsection{数据描述}
本研究所使用的数据集包括2014年到2024年北京市空气质量数据,总计包含11个文件夹,每个文件夹对应一年内的空气质量记录。具体数据内容如下:  
\begin{itemize}
    \item \textbf{文件结构:} 每个文件夹包含逐日空气质量数据(1月1日至12月31日)。
    \item \textbf{主要文件:} 
    \begin{itemize}
        \item \texttt{beijing\_all\_20xxxxxx.csv}:包括PM2.5、PM10、AQI等主要空气质量指标。
        \item \texttt{beijing\_extra\_20xxxxxx.csv}:包括SO2、NO2、CO、O3等附加空气质量指标。
    \end{itemize}
    \item \textbf{数据格式:} CSV 文件。第一行为列名,包括日期、小时、数据类型以及各监测点名。以下每行为某时刻的一种类型数据。
\end{itemize}

\subsection{数据类型}
数据集中所包含的空气质量指标及其单位如表 \ref{tab:datatype} 所示。

\begin{table}[h]
    \centering
    \caption{数据类型及单位}
    \label{tab:datatype}
    \begin{tabular}{|c|l|c|}
        \hline
        \textbf{Type}       & \textbf{数据类型}                        & \textbf{单位}          \\ \hline
        AQI                 & AQI 实时值                              & N/A                    \\ \hline
        PM2.5               & PM2.5 1 小时均值                        & $\mu g/m^3$            \\ \hline
        PM2.5\_24h          & PM2.5 24 小时均值                       & $\mu g/m^3$            \\ \hline
        PM10                & PM10 1 小时均值                         & $\mu g/m^3$            \\ \hline
        PM10\_24h           & PM10 24 小时均值                        & $\mu g/m^3$            \\ \hline
        SO2                 & SO2 1 小时均值                          & $\mu g/m^3$            \\ \hline
        SO2\_24h            & SO2 24 小时均值                         & $\mu g/m^3$            \\ \hline
        NO2                 & NO2 1 小时均值                          & $\mu g/m^3$            \\ \hline
        NO2\_24h            & NO2 24 小时均值                         & $\mu g/m^3$            \\ \hline
        O3                  & O3 1 小时均值                           & $\mu g/m^3$            \\ \hline
        O3\_24h             & O3 24 小时最大值                        & $\mu g/m^3$            \\ \hline
        O3\_8h              & O3 8 小时滑动均值                       & $\mu g/m^3$            \\ \hline
        O3\_8h\_24h         & O3 8 小时滑动均值的 24 小时最大值        & $\mu g/m^3$            \\ \hline
        CO                  & CO 1 小时均值                           & $mg/m^3$               \\ \hline
        CO\_24h             & CO 24 小时均值                          & $mg/m^3$               \\ \hline
    \end{tabular}
\end{table}
\newpage
\subsection{数据样例}
以下是数据文件的样例:  
\begin{verbatim}
date,     hour, type,      东四, 天坛, 官园, 万寿西宫,...
20131205, 1,    PM2.5,     93,   93,   63,  79,...
20131205, 1,    PM2.5_24h, 93,   108,  99,  123,...
20131205, 1,    PM10,      103,  124,  81,  107,...
20131205, 1,    PM10_24h,  97,   130,  122, 141,...
20131205, 1,    AQI,       123,  141,  130, 161,...
\end{verbatim}
\section{数据清洗}
\subsection{清洗结果}  
\texttt{output\_data}中包含了数据的初步清洗和分区整理后得到两个子文件夹:  
\begin{itemize}
    \item \textbf{\texttt{first\_clean}:}  
    该文件夹将同一年的 \texttt{\_all} 数据和 \texttt{\_extra} 数据进行了合并,生成了初始数据集,为后续分析奠定了基础。
    
    \item \textbf{\texttt{by\_district}:}  
    该文件夹下包含以下七个子文件夹,对应 \texttt{AQI, CO, NO2, O3, PM2.5, PM10, SO2} 七类空气质量指标的数据。这些数据按时间维度(逐年、逐月、逐小时)进行了整理,并完成了以下清洗工作:
    \begin{itemize}
        \item 检测并处理缺失值。
        \item 识别并修正异常值。
        \item 按照北京市行政区划(如东城区、西城区等)对监测站点数据进行了分类汇总。
    \end{itemize}
\end{itemize}

\subsection{数据格式}  
清洗后的数据以 CSV 格式存储,文件命名格式为 \texttt{Beijing\_AirQuality\_YYYYXX\_days.csv},其中 \texttt{YYYY} 表示年份,\texttt{XX} 表示空气质量指标(如 \texttt{AQI}、\texttt{PM2.5} 等)。具体数据格式说明如下:  
\begin{itemize}
    \item 第一行为列名,包括日期和各行政区名称(如东城区、西城区等)。
    \item 以下每行为某日期某指标的观测数值。
\end{itemize}

\subsection{数据样例}  
以下是清洗后数据文件的样例 (\texttt{Beijing\_AirQuality\_2014AQI\_days.csv}):  

\begin{verbatim}
date,          东城区,     西城区,      海淀区, ...
20140101,      88.198,     90.625,     105.33, ...
20140102,      119.354,    112.277,    128.236, ...
20140103,      123.979,    123.264,    146.125, ...
20140104,      105.813,    102.916,    104.569, ...
20140105,      172.614,    102.917,    104.569, ...
\end{verbatim}

\newpage
\section{数据可视化}
\subsection{空气治理成效可视化}
\subsubsection{年均AQI逐年变化情况}
\begin{figure}[htb]
    \centering
    \includegraphics[width=\textwidth]{Outputs/北京全年平均AQI趋势图 (2014-2023).png}
    \caption{北京市年均AQI逐年变化情况}
    \label{fig:AQI_yearly}
\end{figure}
图1展示了2014年至2023年北京市全年平均空气质量指数(AQI)的变化趋势。
从图中可以看出,北京市的空气质量在过去十年中呈现出明显的改善趋势。

在2014年,北京市的全年平均AQI为120左右,显示出较高的空气污染水平。
然而,随着时间的推移,AQI值逐年下降,到2018年降至约85。
这一趋势表明北京市在空气质量改善方面取得了显著进展。
尽管在2023年AQI值出现了轻微的波动,但总体上仍保持在较低水平,显示出空气质量的持续改善。
2021年,AQI值下降至约55,达到十年来的最低点。

\subsubsection{主要污染物的逐年变化情况}
\begin{figure}[htb]
    \centering
    \includegraphics[width=\textwidth]{Outputs/主要污染物的逐年变化情况.png}
    \caption{北京市主要污染物浓度逐年变化情况}
    \label{fig:pollutants_yearly}
\end{figure}
图2详细描绘了2015年至2023年北京市六种主要空气污染物浓度相对于前一年变化的百分比。
这些污染物包括一氧化碳(CO)、二氧化氮(NO2)、臭氧(O3)、可吸入颗粒物(PM10)、
细颗粒物(PM2.5)和二氧化硫(SO2)。图中的数据并非直接表示污染物的浓度,
而是展示了每种污染物浓度相对于前一年的变化百分比,
这有助于更直观地理解污染物浓度的增减趋势。

从图中可以观察到,臭氧(O3)的变化百分比相对稳定,
二氧化硫(SO2)的变化百分比几乎一直保持为负值,表明其浓度持续降低,
而其他污染物也在大多年份表现出负的百分变化,说明了其浓度几乎一直在降低,
反映了北京市空气质量的持续改善。

\subsubsection{近年来全年空气质量情况}
\begin{figure}[htb]
    \centering
    \includegraphics[width=\textwidth]{Outputs/2016、2018、2020与2023年日均 AQI 热力图.png}
    \caption{近年来北京市全年空气质量情况}
    \label{fig:air_quality_yearly}
\end{figure}

图3展示了2016年、2018年、2020年和2023年北京市全年空气质量情况的日均AQI热力图。
图中使用了从浅绿色到深蓝色的渐变色来表示不同的AQI值。浅绿色代表较低的AQI值,而深蓝色则代表较高的AQI值。
横轴表示月份,从一月到十二月;纵轴表示一周中的每一天,从星期一到星期日。
并且图上展示了展示了四个不同年份(2016年、2018年、2020年和2023年)的空气质量数据,便于观察和比较不同年份之间的空气质量变化。

从图中可以看出,北京市的空气质量在不同季节存在明显差异。通常,冬季(特别是12月和1月)的空气质量较差,这可能与供暖季节的燃煤排放有关。
相比之下,夏季(特别是6月、7月和8月)的空气质量较好,这可能与较高的温度和较强的大气扩散条件有关。

通过比较不同年份的数据,可以观察到北京市空气质量的改善趋势。例如,
2023年的热力图显示出更多的浅绿色区域,表明这一年的空气质量整体上有所改善。
2016年的图中深蓝色区域较多,表明那一年的空气质量相对较差。随着时间的推移,
深蓝色区域逐渐减少,显示出空气质量的持续改善。

图中还显示出一周内空气质量的变化。通常,周末的空气质量相对较好,这可能与工作日的交通排放和工业活动有关。

\subsection{北京市空气质量的时空特征}
\subsubsection{AQI的分布曲线}
\begin{figure}
    \centering
    \includegraphics[width=0.9\textwidth]{Outputs/AQI的分布曲线.png}
    \caption{北京市 AQI 分布曲线}
    \label{fig:AQI_distribution}
\end{figure}

图4展示了北京市空气质量指数(AQI)的分布曲线。
该图是一个直方图,横轴表示AQI值,纵轴表示频数,即每个AQI区间内出现的次数。

图中的分布曲线呈现出右偏的形态,即大部分数据集中在较低的AQI值区域,
随着AQI值的增加,频数迅速下降。
这种分布形态表明北京市的空气质量大多数时间处于较好的状态,高AQI值的情况较少。

\subsubsection{北京市空气质量的区域特征}

图5是一个箱线图,
展示了2014年至2023年北京市不同区域的空气质量指数(AQI)分布情况。
图中展示了北京市多个区域的AQI分布情况,包括东城区、西城区、朝阳区、
海淀区、丰台区、石景山区、房山区、大兴区、通州区、顺义区、昌平区、
门头沟区、平谷区、怀柔区、密云区、延庆区以及全市平均(mean\_aqi)。
每个区域的AQI分布通过一个箱线图表示,可以直观地比较不同区域之间的空气质量差异。


\begin{figure}[htb]
    \centering
    \includegraphics[width=0.9\textwidth]{Outputs/北京不同区域的AQI箱线图 (2014-2023).png}
    \caption{北京市不同区域AQI箱线图}
    \label{fig:air_quality_district}
\end{figure}

\FloatBarrier

图6是一个柱状图,展示了2014年至2023年北京市各区域强污染的天数。
图中每个柱子代表一个区域,柱子的高度表示该区域在这段时间内AQI大于300的天数,
即强污染天数。图中还标出了平均值为51.75天的参考线,用蓝色虚线表示。

从图中可以看出,不同区域的强污染天数存在显著差异。一些区域的柱状较高,表明这些区域在2014年至2023年间经历了更多的强污染天数。
例如,大兴区、通州区和房山区的柱状明显高于其他区域,表明这两个区域的强污染天数较多。推测可能与当地的产业结构相关。
\begin{figure}[ht]
    \centering
    \includegraphics[width=0.9\textwidth]{Outputs/北京各区域强污染的天数 (2014-2023).png}
    \caption{北京市各区域强污染的天数}
    \label{fig:air_quality_district_heatmap}
\end{figure}

\FloatBarrier

\subsubsection{北京市空气质量的时间特征}
图7展示了一年四季中不同季节空气质量指数(AQI)
在一天内的变化情况。图中用不同颜色的线条分别表示了春季(Spring)、夏季(Summer)、
秋季(Autumn)和冬季(Winter)的AQI变化趋势。

从图中可以看出:春季和冬季的AQI值在一天中的变化趋势相似,均在早晨和晚上较高,而在下午时段较低。
夏季的AQI值在一天中的变化较为平稳,但在下午时段略有上升。
秋季的AQI值在一天中的变化趋势与春季和冬季相似,但在下午时段的下降幅度较小。

所有季节的AQI值在早晨时段(大约5点到8点)达到较高水平,这可能与交通高峰期的排放有关。
在下午时段(大约14点到16点),除了夏季外,其他季节的AQI值都有所下降,这可能与日间的大气扩散条件较好有关。
晚上时段(大约20点以后),AQI值再次上升,可能与夜间供暖、烹饪等活动有关。
\begin{figure}
    \centering
    \includegraphics[width=\textwidth]{Outputs/AQI一日内变化.png}
    \caption{AQI一日内变化}
    \label{fig:air_quality_time}
\end{figure}
\FloatBarrier
\subsubsection{北京市空气质量的时空相关性}
\begin{figure}[ht]
    \centering
    \includegraphics[width=\textwidth]{Outputs/AQI关于城区和月份的聚类热力图.png}
    \caption{北京市空气质量的时空聚类热力图}
    \label{fig:air_quality_correlation}
\end{figure}
图8是一个时空聚类热力图,
展示了北京市不同区域在不同月份的空气质量情况。热力图通过颜色的深浅来表示空气质量指数
(AQI)的高低,颜色越深表示AQI值越高,即空气质量越差。

从图中可以看出,不同区域和不同月份的空气质量存在显著差异。而且根据聚类结果,
我们可以大致得到和之前类似的结论:夏季的空气质量相对较好,冬季的空气质量相对较差;
中心城区的空气质量相对较好,郊区的空气质量相对较差。而且我们还能够看出,
具体每一个城区,在哪些月份的空气质量相对较差,这对于未来的空气质量治理有一定的参考意义。

\subsubsection{AQI与主要污染物的相关关系}
\begin{figure}[ht]
    \centering
    \includegraphics[width=\textwidth]{Outputs/主要污染物以及AQI之间的相互关系.png}
    \caption{AQI与主要污染物的相关关系}
    \label{fig:air_quality_pollutants}
\end{figure}
图9是一个散点图矩阵,展示了空气质量指数(AQI)
与不同主要污染物之间的相关关系。这种图表形式有助于观察多个变量之间的两两关系,
包括它们之间的相关性和分布特征。

对角线上的每个图是一个变量的直方图,展示了该变量的分布情况。
例如,AQI的直方图显示了AQI值的分布,其他污染物(如CO、NO2、O3等)的直方图也类似。
非对角线上的每个图是两个变量之间的散点图,展示了这两个变量之间的关系。
例如,AQI与CO之间的散点图显示了这两个变量之间的关系。

通过观察散点图的分布,可以初步判断两个变量之间是否存在相关性。
如果散点图中的点大致沿着一条直线分布,那么这两个变量之间可能存在线性相关性。
例如,AQI与PM2.5之间的散点图显示出明显的正相关趋势,那么可以推断AQI的增加可能与PM2.5浓度的增加有关。

借助该图,我们可以快速识别变量之间的关系,包括相关性、分布特征和异常值。
从而进一步理解空气质量的成因、评估污染物的影响以及制定空气质量管理策略。
\FloatBarrier
\newpage
\section{结论}
本研究通过对北京市近年来空气质量数据的分析,探讨了北京市空气质量的现状和治理成效。
通过数据的处理和可视化,我们发现了一些有趣的现象,例如北京市空气质量的季节性变化,
不同区域空气质量的差异,以及不同污染物之间的相关性。

具体来说,我们得出了以下结论:
\begin{itemize}
    \item 北京市的空气质量在过去十年中呈现出明显的改善趋势,AQI值逐年下降。
    \item 北京市的主要污染物浓度在过去十年中也呈现出下降趋势,尤其是SO2的浓度持续降低。
    \item 北京市的空气质量在不同季节和不同区域存在显著差异,冬季和郊区的空气质量相对较差。
    \item 北京市的空气质量在一天内也存在明显变化,早晨和晚上的AQI值较高,下午较低。
    \item AQI与主要污染物之间存在一定的相关性,例如AQI与PM2.5之间存在明显的正相关趋势。
\end{itemize}
通过对数据的分析,我们可以更好地理解空气质量的时空特征,为未来的空气质量治理提供参考依据。
本报告的一切数据、代码与可视化成果均已同步至个人github平台:
\url{https://github.com/Welldefine/Data-Visualization}
\section{致谢}
感谢北京市环境保护检测中心和王晓磊提供的数据,为本研究提供了重要的数据支持。
感谢北京市政府及相关部门对空气质量治理工作的支持和努力。
感谢数据可视化技术为我们提供了一种直观、有效的分析工具。
\end{document}
